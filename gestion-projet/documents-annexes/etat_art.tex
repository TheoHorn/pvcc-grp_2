\documentclass{article}
\usepackage[utf8]{inputenc}

\begin{document}
\section*{Projet PPII - Etat de l'art}

    Notre étude va se porter sur les cadres existants autour de notre projet, puis ensuite sera étudié différentes applications numériques mettant en oeuvre le sujet du projet. Pour faire une comparaison avec notre future application, un tableau comparatif a été créé. 
\subsection*{Cadres existants :}

Tout d'abord, une aide gouvernementale a permis le financement de jardins partagés. En effet, le plan de relance économique “France Relance” a permis l’allocation de 17 Millions € au soutien de jardins partagés et collectifs. Les projets de jardins partagés doivent remplir plusieurs critères pour pouvoir accéder à cette allocation. Cette mesure concerne uniquement les zones urbaines et périurbaines et est seulement accessible par les collectivités,les bailleurs sociaux et les associations. L’aide est dédiée à des investissements matériels (comme des outils de jardins, des fournitures ou encore des équipements) et immatériels (prestations d'ingénierie ou études des sols). Par exemple, dans les Bouches du Rhône 500 000 euros ont été investi pour les jardins partagés et collectifs. Cette aide gouvernemental est le cadre existant autour de notre projet. Cela nous ammène donc maitenant sur les différentes application numériques déjà en place sur le marché des jardins partagés. Une étude comparatif a été réalisé avec notre futur projet. 


\subsection*{Tableaux comparatif de l'analyse :}


\begin{tabular}{|p{3cm}|p{3cm}|p{3cm}|p{3cm}|p{3cm}|}
    \hline 
    Applications Critères & Adopte ma Tomate & Le potiron & Planter chez nous 
    \\ \hline
    Possibilité de partager un jardin & Partage sous forme d’annonces & Annonces & Annonces 
    \\ \hline
   Possibilité de rejoindre un jardin & Annonces & Annonces & Annonces  
    \\ \hline
   Carte des jardins à proximité & Carte de tous les jardins à proximité & Carte des annonces & Inexistant 
    \\ \hline
   Conseils de jardinage & Recommandation par saison et localisation & Blog avec articles & Sous formes d’articles 
    \\ \hline
    Notifications & Application smartphone et web & Application smartphone et web & Inexistant 
    \\ \hline
    Utilisation professionnelle & Compte professionnel disponible & Inexistant & Inexistant 
    \\ \hline
    Echange de légumes & Inexistant & Choix des légumes à échanger & Propositions sur le site 
    \\ \hline
    Contact avec un jardinier & Envoi de message & Coordonnées du jardinier & Inexistant 
    \\ \hline
   Mise en place d’évènements & Inexistant & Inexistant & Fête des légumes ou marchés 
    \\ \hline
    Prêt de jardin & Inexistant & Inexistant & Mise en relation avec le jardinier 
    \\ \hline
   Forum & Inexistant & Inexistant & Inexistant 
    \\ \hline
    Format d'application ludique & Inexistant & Inexistant & Inexistant 
    \\ \hline
\end{tabular}

\vspace{0.5cm} \\

\begin{tabular}{|p{3cm}|p{3cm}|p{3cm}|p{3cm}|p{3cm}|}
    \hline 
    Applications Critères & Click and Garden & Jardins de Noé & Prêter son jardin
    \\ \hline
    Possibilité de partager un jardin & Partage sous forme d’annonces & Annonces & Annonces sur le site
    \\ \hline
   Possibilité de rejoindre un jardin & Annonces & Annonces & Contact avec le propriétaire  
    \\ \hline
   Carte des jardins à proximité & Carte et filtres sur les régions & Inexistant & Inexistant
    \\ \hline
   Conseils de jardinage & Recherche guidée & Guide complet & Articles pédagogiques
    \\ \hline
    Notifications & Inexistant & Application web & Inexistant 
    \\ \hline
    Utilisation professionnelle & Inexistant & Rubrique "Jardins Professionnels" & Inexistant
    \\ \hline
    Echange de légumes & Inexistant & Inexistant & Fruits et légumes précisés
    \\ \hline
    Contact avec un jardinier & Inexistant & Inexistant & Inexistant  
    \\ \hline
   Mise en place d’évènements & Inexistant & Inexistant & Inexistant
    \\ \hline
    Prêt de jardin & Annonces par régions & Inexistant & Inexistant
    \\ \hline
   Forum & Inexistant & Forum Interactif & Inexistant 
    \\ \hline
    Format d'application ludique & Inexistant & Inexistant & Inexistant 
    \\ \hline
\end{tabular}

\subsection*{Explication sur le choix des critères de comparaison :}
Les critères de comparaison que nous avons sélectionnés reposent tout particulièrement sur les attentes et les besoins des utilisateurs. Certains de ces critères seront indispensables à notre futur application, et il est donc primordial de connaître au mieux les différentes solutions proposées par les applications existantes. On peut citer, comme exemple, la possibilité de partager un jardin ou de rejoindre un jardin partagé. 
\newline
D’autres critères répondent à un besoin de confort d'ergonomie pour l’utilisateur. On peut citer ici, les moyens de contacts des utilisateurs de la plateforme ou encore la disponibilité de conseils ou de forum. De plus, il est aussi particulièrement important de comparer la cible de ces applications. En effet, celles-ci sont généralement proposées à des particuliers, souvent des familles. Cependant, il faut aussi prendre en compte d'autres catégories de personnes qui pourraient être interessé par ce genre d'application. On peut citer que plusieurs de ces applications ont implantée une partie professionnelle. 
\newline
Certains critères sont aussi utiles dans un aspect de différenciation par rapport aux autres plateformes. En effet, en comparant, des critères plus spécifiques et des idées originales auxquelles nous avons pu pensé, cela pourrait nous permettre de nous différencier du marché actuel. Ici, ce sont les critères autour des possibilités d’échanges des légumes, la mise en place d'événements ou encore une application orienté autour d'un système ludique qui sont soulignés. 
\newline
Cette liste de critères nous apporte donc les principales réponses des besoins auxquels nous devons répondre afin de satisfaire les futurs utilisateurs de notre application. Cependant, d’autres critères bien plus généraux sont aussi à prendre en compte. En effet, l’ergonomie générale du site ou la disponibilité de l’application sur tous les supports sont des critères qui ne sont pas spécifiques aux applications dédiées aux jardins partagés mais pour lesquelles nous devons aussi porter une attention particulière. 


\end{document}



