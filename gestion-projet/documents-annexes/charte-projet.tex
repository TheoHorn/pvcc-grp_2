\documentclass{report}
\usepackage[utf8]{inputenc}
\usepackage {xcolor}

\begin{document}
\begin{center}
    \textbf{\color{orange}{\Huge Projet PPII - Charte}}
\end{center}
\vspace{0.7cm}
\section*{\color{orange}Auteurs}
\begin{tabular}{l|p{8cm}}
   \textbf{Nom / mél} & \textbf{Qualité / rôle}  \\
   \hline 
    DEVEAUX Paul & Membre du projet  \\
    paul.deveaux@telecomnancy.eu & \\ \hline
    HORNBERGER Théo & Membre du projet \\
    theo.hornberger@telecomnancy.eu & \\ \hline
    TEMPESTINI Terry & Membre du projet \\
    terry.tempestini@telecomnancy.eu & \\ \hline
    ARIES Lucas & Membre du projet \\
    lucas.aries@telecomnancy.eu & \\ \hline
\end{tabular}

\vspace{0.5cm}

\section*{\color{orange}Historique des modifications et révisions de ce document}
\begin{tabular}{c|p{10cm}}
   \textbf{N° de version} & \textbf{Description et circonstances de la modification}  \\
   \hline 
    V.0.1 & Brouillon: préparation du documents
    \\ \hline
\end{tabular}

\vspace{0.5cm}

\section*{\color{orange}Validations / autorisations}
\begin{tabular}{c|p{4cm}|p{6cm}}
   \textbf{N° de version} & \textbf{Nom / qualité} & \textbf{Commentaires / réserves}
    \\ \hline
\end{tabular}

\newpage

\begin{center}
    \textbf{\color{orange}{\Huge Charte-projet}} \\
    \textbf{\color{blue}{\Large Table des matières}}
\end{center}

\section*{Résumé}
\section*{1/ Cadrage}
\begin{itemize}
    \item La finalité du projet est la réalisation d'une application web original qui va offrir un service en lien avec l’exploitation de jardins partagés ou privés.
    \item Projet a réaliser en première année à Telecom Nancy pour le module PPII (Projet Pluridisciplinaire d'Informatique Intégrative) en groupe de 4 étudiants.
    \item Les objectifs de ce projet, sont de :  
    \begin{itemize}
        \item Mettre en oeuvre la gestion de projet apprise pendant les cours
        \item Apprendre a travailler à plusieurs sur un même projet.
        \item De concevoir une application web possédant une partie base de données, web et algorithmique.
    \end{itemize}

\end{itemize}

\section*{2/ Déroulement du projet}
\begin{itemize}
    \item L'organisation du projet est jalonnée par les différents rendus attendus par l'école, en terme de ressources et de budget, il n'y a aucun temps sur l'emploi du temps consacré à celui-ci ni de budget, il est réalisé sur le temps libre des différents acteurs du projet.
    \item Les principaux évènements importants sont les réunions et les livrables imposés par le sujet tels que 
        \begin{itemize}
        \item Soutenance Gestion-Projet/État de l'art le 22/10/22.
        \item Rendu de projet le 06/01/23.
        \item Soutenance du projet le 14/01/23.
    \end{itemize}
    \item Les principaux risques sont de sous-estimer le temps nécessaire à la réalisation des différentes tâches et par conséquent de ne pas rendre les livrables dans les délais et les utilisateurs malveillants, quant aux opportunités celles-ci sont plutôt nombreuses notamment grâce à la situation économique actuelle.
\end{itemize}

\newpage
\begin{center}
    \textbf{\color{orange}{\Huge Cadrage}} \\
\end{center}


\newpage
\begin{center}
    \textbf{\color{orange}{\Huge Déroulement du projet}} \\
\end{center}
\section*{\color{orange}{\Large Organisation / ressources, budget}}
\begin{itemize}
    \item Le livre de mission du projet est le suivant : \\ 
    \textit{"Votre objectif est, en mettant en œuvre les principes de Gestion de Projet appris et
    ceux en cours d’acquisition dans le cours de gestion de projet et en mobilisant les acquis
    scientifiques et techniques du module CS54, de concevoir et d’implémenter une application
    innovante dédiées à l’optimisation des ressources dans les vergers et potagers du territoire}
    \item Rôles et responsabilités
    \begin{itemize}
        \item Équipe projet - Réalisation des différents livrables exigés.
        \item Équipe enseignante - Client du projet fixe les limites et demande les livrables.
    \end{itemize}
    \item Budget
    \begin{itemize}
        \item Financier 0€.
        \item Temps libre de 4 étudiants.
        \item Corps enseignant lié au module PPII.
    \end{itemize}
    
\end{itemize}
\section*{\color{orange}{\Large Jalons : échéancier / évènements importants }}
Les différents évènements importants sont les réunions qui sont très régulières dues à la réalisation du projet avec les méthodes AGILES, mais aussi les différents livrables exigés qui rythme la vitesse de réalisation du projet
\vspace{0.5cm} \\
La plupart des jalons importants sont répertoriés à l'aide de la todo-list suivante, Il faut noté que les éléments de la partie informatique ne sont pas encore très détaillé.

\begin{tabular}{|p{4cm}|c|c|c|p{3cm}|}
    \hline
    \textbf{Tâche} & \textbf{Pilote} & \textbf{Échéance} & \textbf{Durée} & \textbf{Commentaire}\\
    \hline
    Réfléchir au concept de l'application informatique & Équipe projet & 14/10/22 & 2 jours & Préparation de la réunion N°2\\
    \hline
    Réalisation de l'état de l'art & Terry, Théo & 17/10/22 & 3 jours & Documents LaTeX, analyse existant / concurrence\\
    \hline
    Gestion de projet & Paul, Lucas & 17/10/22 & 3 jours & SWOT, Todo List, charte du projet, cahier des charges \\
    \hline
    Regroupement travaux - Document Rendu 1 & Équipe projet & 20/10/22 & 3 jours & Documents a rendre à 18H \\
    \hline 
    Réalisation diaporama & Équipe projet & 21/10/22 & 2 jours & Réalisation et préparation de la présentation\\
    \hline
    Soutenance N°1 & Équipe projet & 22/10/22 & 0 jours & Samedi matin\\
    \hline 
    Compléter gestion de projet partie informatique & Équipe projet & & & Attente de plus d'informations\\
    \hline
    Rendu de projet & Équipe projet & 06/01/23 & & Gitlab\\
    \hline
    Réalisation diaporama N°2 & Équipe projet & 13/01/23 & 7 jours & Réalisation et préparation de la présentation\\
    \hline
    Soutenance N°2 & Équipe projet & 14/01/23 & & Fin du projet\\
    \hline
    Réflexion sur le projet et son organisation & Équipe projet & x & x & x\\
    \hline
    
\end{tabular}

\section*{\color{orange}{\Large Risques et opportunités}}
Il y a différents risques et opportunités liées au déroulement du projet. \\ 
La principale opportunité pour le déroulement du projet est liée au corps enseignant et à leurs disponibilités pour répondre aux éventuelles questions, ou résoudre d'éventuels problèmes. \\ \\ 
De plus les jardins partagés deviennent de plus en plus courants notamment à cause de l'inflation des prix de l’alimentation qui incite à produire et manger local mais aussi le plan de relance (France Relance) pour les jardins partagés mais aussi l'absence de concurrent national sur le marché. \\  \\ 
Quant au risques possibles au projet ceux-si sont multiples  \\ 
Que ce soit organisationnel avec notamment des dates imposées pour les livrables et donc la possibilité de ne pas réussir à être dans les délais mais aussi le fait qu'un nombre conséquent de concurrent risque de voir le jour. \\ 
De plus le plus grand risque du projet est que celui-ci repose sur la bienveillance des utilisateurs et l'implication de ceux-ci dans leurs jardins.
\end{document}
